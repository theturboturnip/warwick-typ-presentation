\section{Conclusion \& Future Work}
\begin{frame}{Conclusion}
    % \todomark{Conclusion}
    % 8.5Kloc over 164 files, not including external libraries, comments, or blanks.
% 52 lines per file on avg?
% For C/C++, 132/file on average
% For headers, 30/file
% For CUDA, 51/file (probs counts CUDA headers)
% GLSL, 40/file.

% \todomark{All goals acheived!}
% \begin{center}
%     All goals were achieved!
% \end{center}
% \vfill\null
\begin{wideitemize}
    \item Overall, the project was a success.
    \item CUDA is a very intuitive API, especially for those without prior compute experience.
    \item Vulkan requires more heavy lifting, but it seems to have been worth it.
    \item Looking to the games industry for advice in i.e. particle rendering is helpful.
    
    \item For the scientific community to start using Vulkan, simple abstraction layers will be needed.
    \begin{wideitemize}
        \item VTK, a popular visualization library, has a Vulkan branch that seems to be dead.
        \item Datoviz is a new library with Python bindings that renders with Vulkan.
    \end{wideitemize}
    
    \item CUDA-Vulkan interoperability is nice! Resources should be allocated from Vulkan to maintain full control.
\end{wideitemize}
\end{frame}

\begin{frame}{Future Work}
    Simulation
    \begin{wideitemize}
        \item Investigate simulation accuracy and algorithm.
        \item Re-introduce the Poisson accuracy check.
        \item Optimize parallel reductions.
    \end{wideitemize}
    \vfill\null
    Visualization
    \begin{wideitemize}
        \item Investigate colorblindness options.
        \item Better memory allocation, potentially using a helper library.
        \item Run different layer computations in parallel with separate command buffers?
    \end{wideitemize}
\end{frame}